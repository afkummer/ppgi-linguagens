\begin{frame}
   \frametitle{15 Subtyping}
   \begin{itemize}
      \item Tamb�m chamado de \emph{subtype polymorphism};
      \item Caracter�stica presente nas linguagens orientadas a objetos;
      \item C�lculo Lambda simplesmente tipado com subtipos: \Large $\lambda_{<:}$
   \end{itemize}
\end{frame}

\begin{frame}
   \frametitle{15.1 Subsumption}
   \begin{itemize}
      \item Sem subtipos:
         \begin{itemize}
            \item [$\checkmark$] Regras de tipos bastante r�gidas;
            \item [$\checkmark$] Rejei��o de express�es que, aos olhos do programador, s�o bem tipadas.
         \end{itemize}
      \item Exemplo:
   \end{itemize}
   \begin{center}
      $\infer{\Gamma  \vdash t_1 t_2 : T_{11}} {\Gamma  \vdash t_1 : T_{11} \rightarrow T_{12} \qquad \Gamma \vdash t_2 : T_{11}}$ $\qquad \textsc{(T-App)}$ \\
      
      \vspace{0.4in}
      
      $(\lambda{}r\colon\{x:Nat\}. \quad r.x) \quad \{x=0,y=1\}$ \qquad \alt<2>{\textcolor{green}{V�lido}}{Inv�lido?}
   \end{center}   
\end{frame}

\begin{frame}
   \frametitle{15.1 Subsumption}
   \begin{itemize}
      \item Objetivo dos subtipos:
      \begin{itemize}
         \item [$\checkmark$] Refinamento das regras de tipos;
         \item [$\checkmark$] Se S � subtipo de T ($S <: T$), qualquer termo do tipo S pode ser 
                              utilizado no contexto onde T � esperado;
         \item [$\checkmark$] Princ�pio da substitui��o segura (\emph{safe substituition}).                              
      \end{itemize}
   \end{itemize}
   
   \vspace{0.2in}
   
   \begin{block}{Regra \emph{Subsumption}}
      \begin{center}
         $\infer{\Gamma \vdash t : T} {\Gamma \vdash t : S \qquad S <: T}$ $\qquad \textsc{(T-Sub)}$ \\
      \end{center}
   \end{block}
   \small Adaptado de \citep{pierce}.
\end{frame}

\begin{frame}
   \frametitle{15.1 Subsumption}
   \begin{itemize}
      \item Exemplo: $(\lambda{}r\colon\{x:Nat\}. \quad r.x) \quad \{x=0,y=1\}$
      \begin{itemize}
         \setlength\itemsep{0.1in}
         \item [$\checkmark$] Considerando que: $\{x:Nat,y:Nat\} <: \{x:Nat\}$
         \item [$\checkmark$] A regra \textsc{T-Sub} permite a aplica��o pois s�o tipos v�lidos.
      \end{itemize}   
   \end{itemize}   
\end{frame}