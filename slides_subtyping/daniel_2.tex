\begin{frame}
   \frametitle{15.4 The Top and Bottom Types}
   \begin{block}{Formas de subtipos - Top e Bot}
      \footnotesize\emph{Formas sint�ticas:} \\
      \hspace{1.2in} $T ::= ...$ \\
      \hspace{1.5in} $Top$ \\
      \hspace{1.5in} $Bot$ \\
      
      \vspace{0.15in}
      
      \footnotesize\emph{Regras de subtipos:} \\
      \hspace{1.2in} $S <: Top$ $\qquad \textsc{(S-Top)}$ \\
      \vspace{0.1in}      
      \hspace{1.2in} $Bot <: T$ $\qquad \textsc{(S-Bot)}$ \\
      
      \vspace{0.05in}
   \end{block}
   \small Adaptado de \citep{pierce}.
\end{frame}

\begin{frame}
   \frametitle{15.4 The Top and Bottom Types}
   \begin{itemize}
      \setlength\itemsep{0.1in}
      \item Top:
         \begin{itemize}
            \item [$\checkmark$] Elemento \textbf{m�ximo} da rela��o de subtipos;
            \item [$\checkmark$] Equivale ao tipo \emph{Object} das linguagens orientadas a objetos;
            \item [$\checkmark$] Dispositivo t�cnico sofisticado em sistemas que combinam subtipos com poliformismo.
         \end{itemize}

      \item Bot:
         \begin{itemize}
            \item [$\checkmark$] Elemento \textbf{m�nimo} da rela��o de subtipos;
            \item [$\checkmark$] Tipo vazio (n�o existem valores do tipo Bot);
	    \item [$\checkmark$] Muito �til para expressar algumas opera��es que n�o visam retorno de valores, como exce��es, pois:
	    \begin{itemize}
	       \item [--] Permite ao programador definir express�es sem retorno com o tipo Bot;
	       \item [--] Indica ao \emph{typechecker} que a express�o pode ser utilizada com seguran�a em qualquer contexto.
	    \end{itemize}
         \end{itemize}   
   \end{itemize}
\end{frame}

\begin{frame}
   \frametitle{15.4 The Top and Bottom Types}
   \begin{itemize}
      \item Exemplo:
   \end{itemize}
   $\lambda{}x:T.$ \\
   \quad if $<$ valor apropriado para x $>$ then \\
   \quad \quad $<$ calcula o resultado $>$ \\
   \quad else \\
   \quad \quad error
\end{frame}

\begin{frame}
   \frametitle{15.4 The Top and Bottom Types}
   \begin{itemize}
      \item Tipo \texttt{Bot} dificulta a implementa��o;
      \item Mudan�a da regra de tipo da aplica��o:
      \begin{itemize}
         \item [$\checkmark$] \texttt{t1 t2} : 
         \item [$\checkmark$] \texttt{t1} pode ser tanto do tipo seta ($T1 \rightarrow T2$) ou do tipo \texttt{Bot}.
      \end{itemize}
   \end{itemize}
\end{frame}