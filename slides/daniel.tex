\begin{frame}
   \frametitle{8.1 Tipos (Express�es Aritm�ticas Tipadas)}
   \begin{itemize}
      \item Avalia��o de termos sem tipos:
         \begin{itemize}
            \item [$\checkmark$] Resulta em um valor (\texttt{\frenchspacing true}, \texttt{\frenchspacing false} ou \texttt{\frenchspacing nv} [\texttt{\frenchspacing 0} ou \texttt{\frenchspacing succ nv}]).
            \item [$\checkmark$] Ou trava em algum est�gio da avalia��o no qual nenhuma regra de avalia��o se aplica (\texttt{\frenchspacing pred false}).
         \end{itemize}
   \end{itemize}
\end{frame}

\begin{frame}
   \frametitle{8.1 Tipos (Express�es Aritm�ticas Tipadas)}
   \begin{itemize}
      \item Objetivo dos tipos:
         \begin{itemize}
            \item [$\checkmark$] Identificar termos com erros que ir�o ocasionar um travamento antes de avali�-los.
            \item [$\checkmark$] Tipos criados: \texttt{\frenchspacing Bool} (\textit{booleans}) e \texttt{\frenchspacing Nat} (naturais).
            \item [$\checkmark$] De forma est�tica: \texttt{\frenchspacing t : T} (\texttt{\frenchspacing t} possui tipo \texttt{\frenchspacing T}) indica que \texttt{\frenchspacing t} ir� avaliar para um valor de forma correta (sem travamento) dispensando a necessidade de avali�-lo.
            \item [$\checkmark$] An�lise conservadora: usa apenas a informa��o est�tica, n�o permitindo express�es como \texttt{\frenchspacing if true then 0 else false}, mesmo que elas n�o ocasionem um travamento.
         \end{itemize}
   \end{itemize}
\end{frame}

\begin{frame}
   \frametitle{8.2 Rela��o dos tipos}
   \begin{itemize}
      \item Conjunto de regras de infer�ncia que atribuem tipos ao termos, onde \texttt{\frenchspacing t : T} (o termo \texttt{\frenchspacing t} tem tipo \texttt{\frenchspacing T}):
   \end{itemize}
   \begin{block}{Regras de tipos para booleans}
      \begin{align*}
         & \texttt{\frenchspacing true : Bool} & & (\textsc{T-True})\\
         & \\
         & \texttt{\frenchspacing false : Bool} & & (\textsc{T-False})\\
         & \\
         & \mbox{\infer{\texttt{\frenchspacing if t$_1$ then t$_2$ else t$_3$ : T}}{\texttt{\frenchspacing t$_1$ : Bool} & &  \texttt{\frenchspacing t$_2$ : T} &  & \texttt{\frenchspacing t$_3$ : T}}} & & \textsc{(T-If)}
      \end{align*}
   \end{block}
   Adaptado de \citep{pierce}.
\end{frame}

\begin{frame}
   \frametitle{8.2 Rela��o dos tipos}
   \begin{block}{Regras de tipos para n�meros}
      \footnotesize
      \begin{align*}
         & \texttt{\frenchspacing 0 : Nat} & & \textsc{(T-Zero)}\\
         & \\
         & \mbox{\infer{\texttt{\frenchspacing succ t$_1$ : Nat}}{\texttt{\frenchspacing t$_1$ : Nat}}  } & & \textsc{(T-Succ)}\\
         & \\
         & \mbox{\infer{\texttt{\frenchspacing pred t$_1$ : Nat}}{\texttt{\frenchspacing t$_1$ : Nat}}  } & & \textsc{(T-Pred)}\\
         & \\
         & \mbox{\infer{\texttt{\frenchspacing iszero t$_1$ : Bool}}{\texttt{\frenchspacing t$_1$ : Nat}}  } & & \textsc{(T-IsZero)}\\
      \end{align*}
   \end{block}
   \normalsize
   Adaptado de \citep{pierce}.
\end{frame}

\begin{frame}
   \frametitle{8.2 Rela��o dos tipos}
   \begin{itemize}
      \item Deriva��o de tipos:
      \begin{itemize}
         \item [$\checkmark$] �rvores de inst�ncias de tipos.
         \item [$\checkmark$] Exemplo: \texttt{\frenchspacing if iszero 0 then 0 else pred 0}.
      \end{itemize}
   \end{itemize}
\end{frame}

\begin{frame}
   \frametitle{8.2 Rela��o dos tipos}
   \begin{theorem}[\textsc{Teorema da unicidade dos tipos}]
      Indica que, se um termo possui um tipo, esse tipo � �nico e existe apenas uma
      regra de infer�ncia que deriva a sua constru��o.
   \end{theorem}
\end{frame}
