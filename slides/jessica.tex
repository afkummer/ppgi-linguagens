%inicio jessica
\begin{frame}
\frametitle{Tipos de Fun��es}
\begin{itemize}
\item Para construir um tipo que combine booleanos com primitivas do c�lculo lambda � preciso
adicionar uma classifica��o para os termos cuja avalia��o resulta em uma fun��o;
\item A fim de ter certeza de que a fun��o ir� se comportar corretamente quando for chamada, precisamos manter o controle de qual o tipo de argumento que ela espera.
\end{itemize}
\end{frame}

\begin{frame}
\frametitle{Tipos de Funcoes}
\begin{itemize}
\item Para manter esta informa��o, podemos utilizar um novo tipo:
\end{itemize}
\begin{eqnarray*}
    & T ::= \\
    && Bool \\
    && T \to T\\
\end{eqnarray*}
\end{frame}

\begin{frame}
\frametitle{Tipos de Fun��es}
\begin{itemize}
\item Exemplo:
\end{itemize}
\begin{eqnarray*}
    & Bool \to Bool \\
    & (Bool \to Bool) \to (Bool \to Bool) \\
\end{eqnarray*}
\end{frame}

\begin{frame}
\frametitle{Rela��o  de Tipos}
\begin{itemize}
\item Para saber o tipo de uma abstra��o como \textcolor{red}{$"\lambda x.t"$}, precisamos calcular o que acontece quando essa abstra��o � aplicada a algum argumento;
\item Abordagem utilizada agora: anotar a abstra��o com o tipo esperado para seus argumentos.
\item Exemplo: \textcolor{red}{$"\lambda x.t"$} ser� \textcolor{red}{$"\lambda x:T1 .t2"$}
\end{itemize}
\end{frame}

\begin{frame}
\frametitle{Rela��o  de Tipos}
\begin{itemize}
\item Termos podem conter abstra��es aninhadas. Com isso em mente, utilizaremos $\Gamma \vdash t : T$ onde $\Gamma$ � um conjunto com as vari�veis livres de t e seus respectivos tipos. Sendo assim, a regra de tipo para abstra��es ser�:
\end{itemize}
 \begin{center}
{\huge $\frac{\Gamma ,x\ :\ T_1 \vdash t_2\  :\  T_2}{\Gamma\  \vdash\  \lambda x\  :\ T_1 .t_2\  :\  T_1 \to T_2}$}  $\qquad (T-Abs)$
 \end{center}
\end{frame}

\begin{frame}
\frametitle{Rela��o  de Tipos}
\begin{itemize}
\item A regra para vari�vel �:
\end{itemize}
 \begin{center}
{\huge $\frac{x:T\ \in\ \Gamma}{\Gamma\ \vdash\ x\ :\ T}$}  $\qquad (T-Var)$
 \end{center}
 \begin{itemize}
\item A regra para aplica��o �:
\end{itemize}
 \begin{center}
{\huge $\frac{\Gamma\  \vdash\ t_1\ :\ T_{11} \to T_{12} \quad \Gamma\ \vdash\ t_2 : T_{11}}{\Gamma\  \vdash\ t_1\ t_2\ :\ T_{12}}$}  $\qquad (T-App)$
 \end{center}
\end{frame}
%fim jessica
